\documentclass[10pt,a4paper]{article}

\usepackage{fancyhdr}
\usepackage{lastpage}
\usepackage{extramarks}
\usepackage[utf8]{inputenc}
\usepackage[spanish]{babel}
\usepackage{amsmath}
\usepackage{amsfonts}
\usepackage{amssymb}
\usepackage{graphicx}
\usepackage[usenames,dvipsnames]{color}
\usepackage{listings}
\usepackage{courier}
\usepackage{multirow}
\usepackage{hyperref}

\usepackage[left=2cm,right=2cm,top=2cm,bottom=2cm]{geometry}

\title{Probando LaTeX}
\author{Dylan Rodas}
\date{Noviembre 2018}

\newcommand{\horrule}[1]{\rule{\linewidth}{#1}}

\begin{document}

\begin{tabular}{l l}
\multirow{5}{*}{\includegraphics[width=2cm]{Resources/Logo_UNIS.png}} & Universidad del Istmo de Guatemala \\
& Facultad de Ingeniería \\
& Ingeniería en Sistemas y Ciencias de la Computación \\
& Prácticas de Trabajo e Investigación 4 \\
& Dylan Gabriel Rodas Samayoa - \href{mailto:rodas171315@unis.edu.gt}{rodas171315@unis.edu.gt} \\
\end{tabular}
\\\    
	
\begin{center}
\horrule{1pt}
\huge{Pruebas con LaTeX 1} \\
\large{14 de Noviembre, 2018} \\
\horrule{1pt}
\end{center}

\begin{abstract}
En este documento se realizaran pruebas con LaTeX.
\end{abstract}

Hola se\~nor \emph{Tierra} y hola \emph{J\'upiter}. % Esto es un comentario...

\begin{itemize}
\item Maniacal
\item Knight
\item Nine
\end{itemize}

\begin{center}
\includegraphics[width=4cm]{Resources/Tigre.jpg}
\end{center}

\begin{equation}
\alpha + \beta + 1
\end{equation}
\\
Las palabras se separan por uno o m\'as espacios. \\
Los p\'arrafos se separan por uno o m\'as lineas en blanco. \\
Este    es    un    texto    con    muchos    espacios    eliminados.\\
Comillas simple: `texto'.
Comillas dobles: ``texto''. \\
Caracteres \$\%\&\#! \\
$\infty$    % Simbolo de Infinito.

\section*{Tipografía Matemática: Signo Pesos}
% no tan bueno:
Sean a y b distintos n\'umeros
enteros positivos, y digamos
que c = a - b + 1. \\
% mucho mejor:
Sean $a$ y $b$ distintos n\'umeros
enteros positivos, y digamos
que $c = a - b + 1$. \\
\\
Sea $y=mx+b$ \ldots \\
Sea $y = m x + b$ \ldots % Eliminación de espacios.
\\

\section*{Tipografía Matemática: Notación}
$y = c_2 x^2 + c_1 x + c_0$ \\
$F_n = F_n-1 + F_n-2$ \\ % oops!
$F_n = F_{n-1} + F_{n-2}$ \\ % ok!
$\mu = A e^{Q/RT}$ \\
$\Omega = \sum_{k=1}^{n} \omega_k$ \\

\section*{Tipografía Matemática: Ecuaciones}
Las ra\'ices de una ecuaci\'on cuadr\'atica est\'an dadas por:
\begin{equation}
x = \frac{-b \pm \sqrt{b^2 - 4ac}}
{2a}
\end{equation}
donde $a$, $b$ and $c$ son \ldots

\section*{Entornos}
Podemos escribir $ \Omega = \sum_{k=1}^{n} \omega_k $ en nuestro texto, o podemos escribir:
\begin{equation}
\Omega = \sum_{k=1}^{n} \omega_k
\end{equation}
para mostrarlo en un entorno diferente.

\section*{Entornos: Listas}
\begin{itemize} % por vi\~netas
\item Knight
\item Man\'ia
\end{itemize}
\begin{enumerate} % por n\'umeros
\item Knight
\item Man\'ia
\end{enumerate}

\section*{Tipografía Matemática: Ejemplos con amsmath}
\begin{equation*}
\Omega = \sum_{k=1}^{n} \omega_k
\end{equation*}

\begin{equation*} % bad!
min_{x,y} (1-x)^2 + 100(y-x^2)^2
\end{equation*}
\begin{equation*} % good!
\min_{x,y}{(1-x)^2 + 100(y-x^2)^2}
\end{equation*}

\begin{equation*}
\beta_i =
\frac{\operatorname{Cov}(R_i, R_m)}
{\operatorname{Var}(R_m)}
\end{equation*}

\begin{align*}
(x+1)^3 &= (x+1)(x+1)(x+1) \\
&= (x+1)(x^2 + 2x + 1) \\
&= x^3 + 3x^2 + 3x + 1
\end{align*}

\begin{center}
\section*{Resolución Ejercicio de Tipografía \#1}
\end{center}
In March 2006, Congress raised that ceiling an additional \$0.79 trillion to \$8.97 trillion, which is approximately 68\% of GDP. As of October 4, 2008, the ``Emergency Economic Stabilization Act of 2008'' raised the current debt ceiling to \$11.3 trillion.
\\
\\

\begin{center}
\section*{Resolución Ejercicio de Tipografía \#2}
\end{center}
Sean $X_1, X_2, \ldots, X_n$ una secuencia de variables aleatorias independienets e idénticamente distribuidas con $\operatorname{E}[X_i] = \mu$ y $\operatorname{Var}[X_i] = \sigma^2 < \infty$,  y sea
\begin{equation*}
S_n = \frac{1}{n}\sum_{i}^{n} X_i
\end{equation*}
indica su media. Entonces, cuando $n$ tiende al infinito, las variables aleatorias $\sqrt{n}(S_n - \mu)$ convergen en la distribución a una normal $N(0, \sigma^2)$.
\\

\end{document}