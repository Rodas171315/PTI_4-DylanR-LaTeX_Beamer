\documentclass[10pt,a4paper]{article}

\usepackage{fancyhdr}
\usepackage{lastpage}
\usepackage{extramarks}
\usepackage[utf8]{inputenc}
\usepackage[spanish]{babel}
\usepackage{amsmath}
\usepackage{amsfonts}
\usepackage{amssymb}
\usepackage{graphicx}
\usepackage[usenames,dvipsnames]{color}
\usepackage{listings}
\usepackage{courier}
\usepackage{multirow}
\usepackage{hyperref}
\usepackage{natbib}

\usepackage[left=2cm,right=2cm,top=2cm,bottom=2cm]{geometry}

\title{Probando LaTeX}
\author{Dylan Rodas}
\date{Noviembre 2018}

\newcommand{\horrule}[1]{\rule{\linewidth}{#1}}

\begin{document}

\begin{tabular}{l l}
\multirow{5}{*}{\includegraphics[width=2cm]{Resources/Logo_UNIS.png}} & Universidad del Istmo de Guatemala \\
& Facultad de Ingeniería \\
& Ingeniería en Sistemas y Ciencias de la Computación \\
& Prácticas de Trabajo e Investigación 4 \\
& Dylan Gabriel Rodas Samayoa - \href{mailto:rodas171315@unis.edu.gt}{rodas171315@unis.edu.gt} \\
\end{tabular}
\\\    
	
\begin{center}
\horrule{1pt}
\huge{Pruebas con LaTeX 2} \\
\large{16 de Noviembre, 2018} \\
\horrule{1pt}
\end{center}

\begin{abstract}
En este documento se realizaran pruebas con LaTeX.
\end{abstract}

\section*{Gr\'aficos}
\includegraphics[width=0.3\textwidth,angle=270]{Resources/Tigre.jpg}

\section*{Tablas}
\begin{tabular}{|l|r|r|} \hline
Art.   & Cant. & Uni.\$ \\\hline
DVD    &   1   & 19.99 \\
Sonido &   2   & 39.99 \\
Cable  &   3   & 1.99 \\\hline
\end{tabular}
\\

\begin{center}
\section*{Resolución Ejercicio de Documentos Estructurados \#1}
\end{center}
Se puede encontrar la solución en el documento ``Documento-Estructurado.tex", ver ahí. \\
\\

\begin{center}
\section*{Resolución Ejercicio de Colocar Todo Junto \#2}
\end{center}

\begin{figure}
    \centering
    \includegraphics[width=0.3\textwidth,angle=180]{Resources/Tigre.jpg}
    \caption{\label{fig:tigre}Esto es un tigre\ldots.}
\end{figure}

La figura \ref{fig:tigre} nos muestra un tigre \ldots
\\

\citet{Brooks1997Methodology} muestra que \ldots. Evidentemente todos los n\'umeros impares son primos \citep{Jacobson1999Towards}.
\\

Esto \citet{Sutherland2003UNIVAC}, esto \citet{Taylor2003Influence} y esto \citet{Karthik2001Analysis} son referencias que posteriormente serán agregadas automáticamente a mi listado de referencias. Una más por aquí \citet{Smith1990Enabling}.

\bibliography{referencias.bib}
% Como parametro colocar el nombre de mi archivo con referencias.

\bibliographystyle{abbrvnat}
% Estilo de bibliografia, otro estilo es plainnat.

\end{document}